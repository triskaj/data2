\documentclass{article}

\usepackage[czech]{babel}
\usepackage[utf8]{inputenc}
\usepackage{amsthm}
\usepackage{amsmath}

\newtheorem*{corollary}{Důsledek}
\renewcommand*{\proofname}{Důkaz}
\newtheorem{theorem}{Věta}

\theoremstyle {definition}
\newtheorem{definition}{Definice}
\newtheorem{example}{Příklad}
\newtheorem*{remark}{Poznámka}
\newtheorem{exercise}{Cvičení}

\DeclareMathOperator{\Mod}{Mod}

\begin {document}

\section {Funkční závislosti stanovené z dat}
Pro danou relaci $\mathcal D$ chceme najít, co možná nejmenší teorii
$T$ tak, že $\mathcal D\models A \Rightarrow B$ právě, když $T\models
A\Rightarrow B$.

\begin {definition}
  Teorie $T$ se nazývá \underline {báze $\mathcal D$}, pokud pro
  každou $A\Rightarrow B$ platí $\mathcal D\models A \Rightarrow B$
  p.k. $T\models A\Rightarrow B$.
\end {definition}

\begin {remark}
  Bází $\mathcal D$ je obecně hodně. Např. pokud $T$ je báze $\mathcal
  D$ a navíc $T\models A\Rightarrow B$ pro nějakou $A\Rightarrow B
  \not\in T$, pak $T\cup \{A\Rightarrow B\}$ je opět báze.
\end {remark}

Z definice báze je zřejmé, že budeme-li mít dvě báze, budou mít stejné
sémantické důsledky, je proto žádoucí si takový jev pojmenovat.

\begin {definition}
  Teorie $T_1$ a $T_2$ jsou \underline {sémanticky ekvivalentní},
  značeno $T_1\equiv T_2$, jestliže pro libovolnou $A \Rightarrow B$
  platí $T_1 \models A\Rightarrow B$ právě, když $T_2 \models
  A\Rightarrow B$.
\end {definition}

Sémanticky ekvivalentní teorie, pak mají úzký vztah k pojmu model
teorie.

\begin {theorem}[o charakterizaci sémantické ekvivalence]
  Následující tvrzení jsou ekvivalentní:
  \begin {enumerate}
  \item $T_1\equiv T_2$,
  \item $\Mod(T_1)=\Mod(T_2)$,
  \item $\Mod_C(T_1)=\Mod_C(T_2)$,
  \item Pro libovolnou $A\subseteq R$ máme $[A]_{T_1}=[A]_{T_2}$.
  \end {enumerate}
\end {theorem}
\begin {proof}
  $1\Rightarrow 2$: Pro libovolnou $A\Rightarrow B$ máme
  $\Mod(T_1)\models A\Rightarrow B$ p.k. $T_1\models A\Rightarrow B$
  p.k. $T_2\models A\Rightarrow B$ p.k. $\Mod(T_1)\models A\Rightarrow
  B$. \\

  $2\Rightarrow 3$: Speciální případ.\\

  $3\Rightarrow 4$: Stejné uzávěrové systémy mají stejné uzávěrové
  operátory.\\

  $4\Rightarrow 1$: Pro libovolnou $A\Rightarrow B$ máme $T_1\models
  A\Rightarrow B$ p.k. $B\subseteq[A]_{T_1}=[A]_{T_2}$
  p.k. $T_2\models A\Rightarrow B$.
\end {proof}

\begin {corollary}
  Pokud jsou $T_1$ a $T_2$ báze $\mathcal D$, pak $T_1\equiv T_2$.
\end {corollary}

Pro snadnější charakterizaci pravdivosti v relaci si zavedeme
operátor, který bude fungovat podobně jako sémantický uzávěr u
teorie. Nejdříve však definujeme relaci na n-ticích.

\begin{definition}
  Pro $\mathcal{D}\subseteq \prod_{y\in R}D_y$ a $M\subseteq R$
  definujeme $E_{\mathcal{D}}: 2^R \rightarrow 2^{\mathcal{D} \times
    \mathcal{D}}$ předpisem
  $$E_{\mathcal{D}}(M)= \{\langle t, t'\rangle\in \mathcal
  D\times \mathcal D \mid t(M)=t'(M)\}.$$
\end{definition}

\begin{remark}
\begin {itemize}
\item
  Z definice $E_{\mathcal{D}}$ je hned zřejmé, že relace
  $E_{\mathcal{D}}(M)$ je ekvivalencí na $\mathcal{D}$ což také
  znamená, že můžeme udělat rozklad.

\item
  Význam vztahu $E_{\mathcal{D}}(M)\subseteq E_{\mathcal{D}}(M')$ je,
  že všechny dvojce n-tic, které se rovnají na $M$ se také rovnají na
  $M'$.

\item
  $E_{\mathcal{D}}$ je zřejmě antinonní, protože pokud $M_1\subseteq
  M_2$, všechny n-tice, které se rovnají na $M_2$ se tím spíš musí
  rovnat na $M_1$, tedy $E_{\mathcal{D}}(M_2)\subseteq
  E_{\mathcal{D}}(M_1)$.
\end {itemize}
\end{remark}

\begin {definition}
  Pro $\mathcal{D}\subseteq \prod_{y\in R}D_y$ a $M\subseteq R$
  definujeme $C_{\mathcal D}: 2^R \rightarrow 2^R$ předpisem
  $$C_{\mathcal{D}}(M) = \{y \in R \mid E_{\mathcal{D}}(M) \subseteq
  E_{\mathcal{D}}(\{y\})\}.$$
\end {definition}
\begin {remark}
  $C_{\mathcal D}(M)$ je vlastně množina atributů, na kterých jsou si
  rovny všechny dvojice n-tic z $\mathcal{D}$, které jsou si rovny na
  $M$. Důsledkem pak je, že $E_{\mathcal{D}}(M)\subseteq
  E_{\mathcal{D}}(C_{\mathcal D}(M))$. Důkaz je ponechán čtenáři.
\end {remark}

\begin{theorem}
 $C_{\mathcal{D}}$ je uzávěrový operátor na $R$.
\end{theorem}
\begin{proof}
  \begin{itemize}
  \item \emph {(extenzivita)}: Pokud $y \in M$, pak
    $E_{\mathcal{D}}(M) \subseteq E_{\mathcal{D}}(\{y\})$, protože
    pokud jsou si $t, t'$ rovny na všech atributech z $M$, tím spíš
    jsou si rovny na $y \in M$. Odtud dle definice $C_{\mathcal{D}}$
    dostáváme $y \in C_{\mathcal{D}}(M)$.

  \item \emph{(monotonie)}: Předpokládejme $M_1 \subseteq M_2$ a
    vezmeme $y \in C_{\mathcal{D}}(M_1)$. Poslední znamená, že
    $E_{\mathcal{D}}(M) \subseteq E_{\mathcal{D}}(\{y\})$. Z antitonie
    $E_{\mathcal{D}}$ dostáváme $E_{\mathcal{D}}(M_2) \subseteq
    E_{\mathcal{D}}(M_1) \subseteq E_{\mathcal{D}}(\{y\})$. Z definice
    $C_{\mathcal{D}}$ je $y \in C_{\mathcal{D}}(M_2)$.
    
  \item \emph{(idempotence)}: $C_{\mathcal{D}}(M) \subseteq
    C_{\mathcal{D}}(C_{\mathcal{D}}(M))$ platí z extenzivity. Pro
    obrácenou inkluzi máme následující posloupnost argumentů:

    \begin {align*}
      E_{\mathcal{D}}(M) &\subseteq E_{\mathcal{D}}(C_{\mathcal{D}}(M))
      \\
      \{y \in R \mid E_{\mathcal{D}}(C_{\mathcal{D}}(M)) \subseteq
      E_{\mathcal{D}}(\{y\})\} &\subseteq \{y \in R \mid
      E_{\mathcal{D}}(M) \subseteq E_{\mathcal{D}}(\{y\})\}
      \\
      C_{\mathcal{D}}(C_{\mathcal{D}}(M)) &\subseteq
      C_{\mathcal{D}}(M)
    \end {align*}
\end{itemize}
\end{proof}

\begin{theorem}[o charakterizaci pravdivosti]
Následující jsou ekvivalentní:
\begin{enumerate}
\item
$\mathcal{D} \models A \Rightarrow B$
\item
$E_{\mathcal{D}}(A) \subseteq E_{\mathcal{D}}(B)$
\item
$B \subseteq C_{\mathcal{D}}(A)$
\end{enumerate}
\end{theorem}

\begin{proof}
  $1\Rightarrow2$: Z definice $\mathcal{D} \models A \Rightarrow B$,
  pokud $t(A)=t'(A)$, pak $t(B)=t'(B)$, tzn. pokud $\langle t, t'
  \rangle \in E_{\mathcal{D}}(A)$, pak $\langle t, t' \rangle \in
  E_{\mathcal{D}}(B)$, tj. $E_{\mathcal{D}}(A) \subseteq
  E_{\mathcal{D}}(B)$.\\

  $2\Rightarrow3$: Předpokládejme $E_{\mathcal{D}}(A) \subseteq
  E_{\mathcal{D}}(B)$. Pro libovolný $y \in B$ pak z antitonie platí
  $E_{\mathcal{D}}(A) \subseteq E_{\mathcal{D}}(B) \subseteq
  E_{\mathcal{D}}(\{y\})$. To podle definice $C_{\mathcal{D}}$
  znamená, že $y \in C_{\mathcal{D}}(A)$, tj. $B \subseteq
  C_{\mathcal{D}}(A)$.\\

  $3\Rightarrow1$: Předpokládejme $B \subseteq C_{\mathcal{D}}(A)$.
  Dále mějme $t, t'\in \mathcal D$ takové, že $t(A)=t'(A)$ a vezmeme
  libovolné $y \in B$. Pak nutně $\langle t, t' \rangle \in
  E_{\mathcal{D}}(A)$ a navíc $E_{\mathcal{D}}(A) \subseteq
  E_{\mathcal{D}}(\{y\})$. Důsledkem je, že $t(y)=t'(y)$, tedy
  $t(B)=t'(B)$.
\end{proof}

\begin{theorem}[o charakterizaci báze]
  $T$ je báze $\mathcal{D}$ právě, když pro libovolné $A\subseteq R$
  máme $C_{\mathcal{D}}(A)=[A]_T$.
\end{theorem}

\begin{remark}
  Ekvivalentně $C_{\mathcal{D}}(M)=[M]_T = M^{\infty}_T = M^{+}_T$.
\end{remark}

\begin{proof}
  "$\Rightarrow$": Nechť $T$ je báze $\mathcal{D}$. Pak $[M]_T
  \subseteq [M]_T$ p.k. $T \models M \Rightarrow [M]_T$
  p.k. $\mathcal{D} \models M \Rightarrow [M]_T$ p.k. $[M]_T \subseteq
  C_{\mathcal{D}}(M)$. Obráceně máme $C_{\mathcal{D}}(M) \subseteq
  C_{\mathcal{D}}(M)$ p.k. $\mathcal{D} \models M \Rightarrow
  C_{\mathcal{D}}(M)$ p.k. $T \models M \Rightarrow
  C_{\mathcal{D}}(M)$ p.k. $C_{\mathcal{D}}(M) \subseteq [M]_T$.
  Dohromady tedy $C_{\mathcal{D}}(M) =[M]_T$. \\
  
  "$\Leftarrow$": Pokud $C_{\mathcal{D}}$ má stejné pevné body jako
  $[\dots]_T$, pak $\mathcal{D} \models A \Rightarrow B$ p.k. $B
  \subseteq C_{\mathcal{D}}(A) = [A]_T$ p.k. $T \models A \Rightarrow
  B$.
\end{proof}

Následující věta ukazuje, že pro libovolnou relaci existuje minimálně
jedna báze.

\begin{theorem}[o existenci báze]
  $T = \{A \Rightarrow C_{\mathcal{D}}(A) \mid A \subseteq R\}$ je
  báze $\mathcal{D}$.
\end{theorem}

\begin{proof}
  Dle předchozí věty stačí ověřit, že $C_{\mathcal{D}}(M) = [M]_T$ pro
  libovolnou $M$, tzn. ověřit, že $M = C_{\mathcal{D}}(M)$ právě když
  $M \in \mathcal{M}_T$.\\

  "$\Rightarrow$": Předpokládejme $M = C_{\mathcal{D}}(M)$ a vezmeme
  libovolnou $A \Rightarrow C_{\mathcal{D}}(A) \in T$ tak, že $A
  \subseteq M$. Z monotonie operátoru $C_{\mathcal{D}}$ dostaneme
  $C_{\mathcal{D}}(A) \subseteq C_{\mathcal{D}}(M) = M$. Dohromady
  tedy $\mathcal{D}_M \models A \Rightarrow C_{\mathcal{D}}(A)$
  p.k. $\mathcal{D}_M \in \Mod_C(T)$ p.k. $M \in \mathcal{M}_T$.\\

  "$\Leftarrow$": Předpokládejme, že $M \in \mathcal{M}_T$. To jest
  $\mathcal{D}_M \in \Mod_C(T)$. Speciálně pro $M \Rightarrow
  C_{\mathcal{D}}(M) \in T$ máme $\mathcal{D}_M \models M \Rightarrow
  C_{\mathcal{D}}(M)$. Odtud $C_{\mathcal{D}}(M) \subseteq M$ a
  přidáme-li extenzivitu $C_{\mathcal{D}}$ dostaneme
  $C_{\mathcal{D}}(M) = M$.
\end{proof}

Když už víme, že báze vždy existuje, přesuneme pozornost na její
velikost vzhledem k počtu FZ. Z předchozího textu vyplývá, že se
budeme snažit najít bázi ekvivalentní, ale co nejmenší.

První ideou je ostranit z teorie nějaké FZ tak, že je pořád
bází. Pokud už nejde zmenšit je tzv. neredundantní.

\begin{definition}
  Teorie $T$ je \underline{neredundantní báze} relace $\mathcal{D}$,
  pokud $T$ je báze $\mathcal{D}$ a pro každou $T' \subset T$ platí,
  že $T'$ není báze $\mathcal{D}$.
\end{definition}

Tato definice má i ekvivalentní formulaci, která vede na jednoduchý
algoritmus transformace báze na bázi neredundandní.

\begin{theorem}[o charakterizaci neredundantní báze]
  Teorie $T$ je neredundantní báze relace $\mathcal{D}$ právě, když
  $T$ je báze a žádná $A \Rightarrow B \in T$ sémanticky neplyne z
  $T \setminus \{A \Rightarrow B \}$.
\end{theorem}

\begin{proof}
  "$\Rightarrow$": Vezmeme libovolnou $A \Rightarrow B \in T$. Z
  předpokladu, že $T$ je báze, vyplývá, že $T \setminus \{A
  \Rightarrow B \}$ není báze, a tedy není ekvivalentní $T$. Pak
  existuje model $T \setminus \{A \Rightarrow B \}$, který není
  modelem $T$, a tedy není v něm pravdivá $A\Rightarrow B$, což
  znamená, že $T \setminus \{A \Rightarrow B \} \not\models A
  \Rightarrow B$.\\
  
  "$\Leftarrow$": Vezmeme libovolnou $T' \subset T$. Pak nutně
  existuje $A \Rightarrow B \in T$ tak, že $A \Rightarrow B \notin
  T'$, a díky předpokladu máme $T' \not\models A \Rightarrow B$. Tudíž
  $T$ a $T'$ nejsou sémanticky ekvivalentní.
\end{proof}

K tomu, abychom definovali konkrétní neredundantní bázi využijeme
následující množinu.

\begin{definition}
  Pro relaci $\mathcal{D}$ nad $R$ uvažujeme množinu $P_{\mathcal{D}}
  \subseteq 2^R$, která je definovaná předpisem:
  
  $$ \mathcal P_{\mathcal{D}} = \{ P \neq C_{\mathcal{D}}(P) \mid
  \forall Q \in \mathcal P_{\mathcal{D}}: \text{ pokud } Q \subset P
  \text{, pak } C_{\mathcal{D}}(Q) \subseteq P \}. $$
\end{definition}

Prvkům z této množiny se někdy říká pseudo-uzávěry. Koresponduje to s
tím, že to nejsou sice uzavřené množiny, ale mají uzávěrovou vlastnost
vzhledem ke všem ostatním prvkům z této množiny.

\begin {definition}
  \underline {GD bází} $\mathcal D$ nazveme teorii definovanou
  následujícím předpisem:
  $$GD(\mathcal D)= \{P \Rightarrow C_{\mathcal{D}}(P) \mid P \in
  \mathcal P_{\mathcal{D}}\}.$$
\end {definition}

\begin {remark}
  Teorie je nazvaná podle francouzkých vědců Guigues a Duquenne, kteří
  ji prvně definovali v kontextu formální konceptuální analýzy.
\end {remark}

\begin{theorem}
  GD báze relace $\mathcal D$ je bází $\mathcal D$.
\end{theorem}
\begin{proof}
  $GD(\mathcal D)$ je báze právě, když $C_{\mathcal{D}}$ a
  $[\dots]_{GD(\mathcal D)}$ mají stejné pevné body, tedy $M =
  C_{\mathcal{D}}(M)$ právě, když $M \in \mathcal{M}_{GD(\mathcal D)}$
  pro každou $M\subseteq R$.\\
  
  "$\Rightarrow$": Víme, že $T' = \{A \Rightarrow C_{\mathcal{D}}(A)
  \mid A \subseteq R\}$ je báze a vidíme, že $T \subseteq T'$, z čehož
  vyplývá, že $\mathcal{M}_{T'} \subseteq \mathcal{M}_T$. Nechť $M =
  C_{\mathcal{D}}(M)$. Pak $M \in \mathcal{M}_{T'}$, a tedy z
  předchozího $M \in \mathcal{M}_T$.\\
  
  "$\Leftarrow$": Nechť $M \in \mathcal{M}_T$, pak $\mathcal{D}_M
  \models P \Rightarrow C_{\mathcal{D}}(P)$ pro každou $P \in
  \mathcal P_{\mathcal{D}}$, což znamená, že pokud $P \subseteq M$, pak
  $C_{\mathcal{D}}(P) \subseteq M$. Nyní sporem dokážeme, že $M =
  C_{\mathcal{D}}(M)$. Nechť tedy $M \neq C_{\mathcal{D}}(M)$. Pak
  díky předchozímu je $M \in \mathcal P_{\mathcal{D}}$, a tedy z předpokladu
  plyne, že $\mathcal{D}_M \models M \Rightarrow
  C_{\mathcal{D}}(M)$. Jelikož ale $M \subseteq M$, pak
  $C_{\mathcal{D}}(M) \subseteq M$. Navíc z extenzivity uzávěrového
  operátoru $C_{\mathcal{D}}$ máme $M \subseteq C_{\mathcal{D}}(M)$,
  což dohromady dává $M = C_{\mathcal{D}}(M)$.
\end{proof}

\begin {theorem}
  GD báze relace $\mathcal D$ je neredundantní bází $\mathcal D$.
\end {theorem}
\begin {proof}
  Vezmeme libovolnou $T\subset GD(\mathcal D)$ a ukážeme, že $T$ není
  báze. Díky předpokladu existuje $P \Rightarrow C_{\mathcal{D}}(P)\in
  GD(\mathcal D)$, která není v $T$. Z definice $\mathcal
  P_{\mathcal{D}}$ je vidět, že $\mathcal{D}_P$ je modelem
  $T$. Jelikož však není modelem $GD(\mathcal D)$, nemohou být $T$ a
  $GD(\mathcal D)$ sémanticky ekvivalentní, což dohromady s tím, že
  $GD(\mathcal D)$ je báze $\mathcal D$ dává, že $T$ není báze
  $\mathcal D$.
\end {proof}

Jako motivaci pro zbytek sekce vezmeme následující příklad.

\begin {example}
  Mějme $R=\{a,b,c\}$ a teorie $T_1=\{\{a\}\Rightarrow \{b,c\}\}$ a
  $T_2=\{\{a\}\Rightarrow \{b\}, \{a\}\Rightarrow \{b\}\}$. Při
  bližším prozkoumání zjistíme, že $T_1\equiv T_2$ a navíc, že jsou
  obě neredundantní, ale $|T_1| < |T_2|$.
\end {example}

Z příkladu je patrné, že kompaktnější verze teorií nelze získat pouze
ostraněním redundantních FZ. Jelikož chceme najít teorii s nejmenší
kardinalitou, definujeme následující pojem.

\begin{definition}
  Pokud teorie $T$ je báze $\mathcal{D}$ a pro libovolnou $T'$, která je také
  bází $\mathcal{D}$, platí, že $|T| \leq |T'|$, pak se $T$ nazývá
  \underline{minimální báze} $\mathcal{D}$.
\end{definition}

Ukazuje se, že GD báze je také minimální bazí, ale abychom byli
schopni to potvrdit, je potřeba prozkoumat vzájemné vlastnosti mezi
množinou $\mathcal P_{\mathcal{D}}$ a operátorem $C_{\mathcal{D}}$.

\begin {remark}
  Fakt, že $|T| \leq |T'|$ intuitivně chápeme, ale přesnou
  matematickou definicí je, že existuje injektivní zobrazení $f:T
  \rightarrow T'$. Pro připomenutí, zobrazení $f$ je injektivní,
  jestliže pro každé $x_1,x_2\in T$ platí, že pokud $f(x_1)=f(x_2)$,
  pak $x_1=x_2$, nebo-li neexistují dva prvky, které se zobrazí na to
  samé, a tedy pro konečné množiny platí, že $T$ má menší nebo stejný
  počet prvků jako $T'$. Této definice využijeme u důkazu, že GD báze
  je minimální.
\end {remark}

Jednou ze základních vlastností pseudo-uzávěrů je, že přidáme-li jeden
z nich do uzávěrového systému generovaným operátorem
$C_{\mathcal{D}}$, pak je výsledná množina zase uzávěrovým systémem.

\begin{theorem}
  Pokud $P \in \mathcal P_{\mathcal{D}}$ a $A \subseteq R$ tak, že $P
  \not\subseteq C_{\mathcal{D}}(A)$, pak $C_{\mathcal{D}}(A) \cap P =
  C_{\mathcal{D}}(C_{\mathcal{D}}(A) \cap P)$.
\end{theorem}

\begin{proof}
  Předpokládejme, že $P \not\subseteq C_{\mathcal{D}}(A)$ a tedy $P
  \not\subseteq C_{\mathcal{D}}(A) \cap P$. Nyní stačí ukázat, že
  $\mathcal{D}_{C_{\mathcal{D}}(A) \cap P}$ je modelem $T$, z čehož
  pak plyne $C_{\mathcal{D}}(A) \cap P \in \mathcal{M}_T$ a tedy, že
  $C_{\mathcal{D}}(A) \cap P = C_{\mathcal{D}}(C_{\mathcal{D}}(A) \cap
  P)$.

  Vezmeme libovolnou $Q \in \mathcal P_{\mathcal{D}}$ tak, že $Q
  \subseteq C_{\mathcal{D}}(A) \cap P$ a prokážeme, že
  $C_{\mathcal{D}}(Q) \subseteq C_{\mathcal{D}}(A) \cap P$. K tomu nám
  postačí fakt, že $C_{\mathcal{D}}(Q)$ je podmnožinou obou množin. Z
  předpokladů $Q \subseteq C_{\mathcal{D}}(A) \cap P$ a $P
  \not\subseteq C_{\mathcal{D}}(A)$ nutně plyne, že $Q \subset P$, a
  tedy z definice $\mathcal P_{\mathcal{D}}$ pak $C_{\mathcal{D}}(Q)
  \subseteq P$. Druhý fakt je už jednoduchý, jelikož z předpokladu
  nutně plyne $Q \subseteq C_{\mathcal{D}}(A)$ a tedy využitím
  monotonie a idempotence operátoru $C_{\mathcal{D}}$ dostaneme
  $C_{\mathcal{D}}(Q) \subseteq C_{\mathcal{D}}(C_{\mathcal{D}}(A)) =
  C_{\mathcal{D}}(A)$.
\end{proof}

Díky této vlastnosti můžeme dát do korespondence FZ GD báze a
libovolné jiné báze. Přesněji, zaručí nám existenci injektivního
zobrazení z množiny $\mathcal P_{\mathcal{D}}$ do libovolné báze.

\begin{theorem}
  Nechť $T$ je báze $\mathcal{D}$. Potom pro každou $P \in \mathcal
  P_{\mathcal{D}}$ existuje $A \Rightarrow B \in T$ taková, že
  $C_{\mathcal{D}}(A) = C_{\mathcal{D}}(P)$ a $\mathcal{D}_P
  \not\models A \Rightarrow B$.
\end{theorem}
\begin{proof}
  Nechť $P \in \mathcal P_{\mathcal{D}}$. Z toho plyne, že $P \neq
  C_{\mathcal{D}}(P)$ a jelikož $T$ je báze, máme taky $P \not\in
  \mathcal{M}_T$, tedy existuje $A \Rightarrow B \in T$ tak, že
  $\mathcal{D}_P \not\models A \Rightarrow B$. Nyní ukážeme, že
  $C_{\mathcal{D}}(A) = C_{\mathcal{D}}(P)$. \\
  
  "$\subseteq$": Jestliže $\mathcal{D}_P \not\models A \Rightarrow B$,
  pak nutně $A\subseteq P$ a zbytek plyne z monotonie operátoru
  $C_{\mathcal{D}}$. \\

  "$\supseteq$": Stačí dokázat, že $P\subseteq C_{\mathcal D}(A)$,
  protože zbytek plyne z monotonie a idempotence operátoru
  $C_{\mathcal D}$. Tvrzení dokážeme sporem, tedy předpokládejme, že
  platí $P\not\subseteq C_{\mathcal D}(A)$. Jelikož $A\Rightarrow B\in
  T$, pak $T\models A\Rightarrow B$, dále pak $B\subseteq
  C_{\mathcal{D}}(A) = [A]_T$, protože $T$ je báze. Dalším je, že
  $B\not\subseteq P$, protože $\mathcal{D}_P \not\models A \Rightarrow
  B$. Dohromady to znamená, že $C_{\mathcal{D}}(A) \not\subseteq P$, a
  když k tomu přidáme ještě předpoklad $P\not\subseteq C_{\mathcal
    D}(A)$ zjistíme, že $C_{\mathcal D}(A) \cap P \subset C_{\mathcal
    D}(A)$. Ze stejného předpokladu máme $A\subseteq P$, tedy
  $C_{\mathcal D}(A)\subseteq C_{\mathcal D}(P)$. Navíc $A \subseteq
  C_{\mathcal D}(A)$, což s předchozím dává $A\subseteq C_{\mathcal
    D}(A) \cap P$. Monotonií dostaneme $C_{\mathcal D}(A)\subseteq
  C_{\mathcal D}(C_{\mathcal D}(A) \cap P)$ a díky předchozí větě
  $C_{\mathcal D}(A)\subseteq C_{\mathcal D}(A) \cap P$, což je ale v
  rozporu s tím, že $C_{\mathcal D}(A) \cap P \subset C_{\mathcal
    D}(A)$.
\end{proof}

Další vlastnost koresponduje s tou, kterou jsme nazvali
základní. Přidáme-li do uzávěrového systému některé dva
pseudo-uzávěry, zůstane pořád uzávěrovým systémem.

\begin{theorem}
  Pokud $P_1, P_2 \in \mathcal P_{\mathcal{D}}$, $P_1 \not\subseteq
  P_2$ a $P_2 \not\subseteq P_1$, pak $C_{\mathcal{D}} (P_1 \cap P_2)
  = P_1 \cap P_2$.
\end{theorem}

\begin{proof}
  Nejdříve položme 
  \begin {align*}
  T_1 &= GD(\mathcal D) \setminus \{P_1 \Rightarrow C_{\mathcal{D}}(P_1)\}, \\
  T_2 &= GD(\mathcal D) \setminus \{P_2 \Rightarrow C_{\mathcal{D}}(P_2)\}.
  \end {align*}
  Pak z definice $\mathcal P_{\mathcal D}$ máme, že
  $\mathcal{D}_{P_1}$ je modelem $T_1$ a $\mathcal{D}_{P_2}$ je
  modelem $T_2$. Tím spíš jsou pak obě relace modely $T_1\cap
  T_2$. Tím pádem i $\mathcal{D}_{P_1 \cap P_2}$ musí být model $T_1
  \cap T_2$, protože $\mathcal{M}_{T_1 \cap T_2}$, je uzávěrový
  systém. Navíc $\mathcal{D}_{P_1 \cap P_2}$ je i modelem $T_1$,
  protože z předpokladu $P_2 \not\subseteq P_1$ plyne $P_2
  \not\subseteq P_1\cap P_2$. To samé platí i pro $T_2$, tedy
  $\mathcal{D}_{P_1 \cap P_2}$ je modelem $T_1 \cup T_2 = GD(\mathcal
  D)$. To znamená, že $P_1 \cap P_2 = [P_1 \cap P_2]_{GD(\mathcal
    D)}$, a jelikož $GD(\mathcal D)$ je navíc báze $\mathcal D$, máme
  $[P_1 \cap P_2]_{GD(\mathcal D)} = C_{\mathcal D}(P_1 \cap
  P_2)$. Dohromady tedy $C_{\mathcal{D}}(P_1 \cap P_2) = P_1 \cap
  P_2$.
\end{proof}

Než přistoupíme k důkazu, že GD báze je minimální vzhledem k počtu FZ,
je potřeba se zamyslet Použitím předchozích vlastností můžeme dokázat
následující tvrzení.

\begin{theorem}
  GD báze $\mathcal{D}$ je minimální báze $\mathcal{D}$.
\end{theorem}

\begin{proof}
  K prokázání tvrzení nám stačí pro libovolnou bázi $\mathcal{D}$,
  označme ji $T$, najít injektivní zobrazení $f : \mathcal P_{\mathcal{D}}
  \rightarrow T$. Z předchozích tvrzení víme, že pro každou $P \in
  \mathcal P_{\mathcal{D}}$ existuje $A \Rightarrow B \in T$ tak, že platí
  $C_{\mathcal{D}}(A) = C_{\mathcal{D}}(P)$ a $\mathcal{D}_P
  \not\models A \Rightarrow B$. Položme tedy $f(P)$ rovno takovéto
  $A\Rightarrow B$ a ukážeme, že se jedná o injektivní zobrazení.

  Nechť $P_1$ a $P_2$ jsou prvky $\mathcal P_{\mathcal D}$ takové, že
  $f(P_1) = f(P_2)$. Z předchozí věty jsou tyto obrazy podle $f$ rovny
  $A\Rightarrow B\in T$ a platí, že $C_{\mathcal{D}}(P_1) =
  C_{\mathcal{D}}(A) = C_{\mathcal{D}}(P_2)$. Nyní není možné, aby
  $P_1 \subset P_2$. Vskutku, kdyby to tak bylo, dle definice
  $\mathcal P_{\mathcal D}$ bychom měli $C_{\mathcal{D}}(P_1)
  \subseteq P_2$ a $P_2\neq C_{\mathcal{D}}(P_2)$, což spolu s
  extenzivitou $C_{\mathcal D}$ dává $C_{\mathcal{D}}(P_1)\subset
  C_{\mathcal{D}}(P_2)$. To je však v rozporu s předchozím. Stejně tak
  nemůže nastat $P_2 \subset P_1$. Navíc také nemůže nastat $P_2 \neq
  P_1$, jinak bychom měli spor s $C_{\mathcal{D}}(P_1) =
  C_{\mathcal{D}}(A)$, protože dle předchozí věty by pak
  $C_{\mathcal{D}}(P_1\cap P_2) = P_1\cap P_2$ a spolu s definicí $f$
  bychom měli $A\subseteq P_1$ a $A\subseteq P_2$, tedy $A\subseteq
  P_1\cap P_2$, což by pak použitím monotonie operátoru
  $C_{\mathcal{D}}$ dalo $C_{\mathcal{D}}(A)\subseteq P_1\cap
  P_2\subset C_{\mathcal{D}}(P_1)$. Jedinná možnost je tedy, že $P_1 =
  P_2$, čímž jsme prokázali injektivitu $f$.
\end{proof}

Jedinnou otázkou teď je, jak takovouto bázi najít. K tomu nám bude
sloužit jemně upravený operátor $\dots^{\infty}_{T}$.

\begin {definition}
  Pro $M\subseteq R$ zavedeme následující posloupnost podmnožin
  $R$:
  \begin {align*}
    M^{(0)}_{GD(\mathcal{D})} &= M,\\
    M^{(i+1)}_{GD(\mathcal{D})} &=
    \bigcup\{B\mid A\Rightarrow B\in GD(\mathcal{D}) \text { a }
    A\subset M^{(i)}_{GD(\mathcal{D})}\},\\
    M^{(\infty)}_{GD(\mathcal{D})} &=
    \bigcup_{i=0}^{\infty}M^{(i)}_{GD(\mathcal{D})}.
  \end {align*}
\end {definition}

I v tomto případě je jasné, že $M^{(\infty)}_{GD(\mathcal{D})}$ je
konečná, protože $R$ je konečná a díky tomu se růst posloupnosti někdy
zastaví.

\begin {exercise}
  \begin {enumerate}
  \item Dokažte, že $\dots^{(\infty)}_{GD(\mathcal{D})}$ je uzávěrový
    operátor na $R$.
  \item Pro každou množinu $M\subseteq R= \{a,b,c,d,e,f,g,h\}$ a
    teorii
    \begin {align*}
      T= \{
      &\{a,b\}\Rightarrow \{c\},\\
      &\{b\}\Rightarrow \{d\},\\
      &\{c,d\}\Rightarrow \{e\},\\
      &\{c,e\}\Rightarrow \{g,h\},\\
      &\{g\}\Rightarrow \{a\}\}
    \end {align*}
    vypočtěte $M^{(\infty)}_{GD(\mathcal{D})}$.
  \item Dokažte, že platí $M^{(\infty)}_{GD(\mathcal{D})} = M$ právě, když
    platí buď $M = C_{\mathcal{D}}(M)$ nebo $M\in \mathcal P_{\mathcal D}$.
  \item Naprogramujte operátor $\dots^{(\infty)}_{GD(\mathcal{D})}$.
    (hint: inspirujte se algoritmem \texttt {CLOSURE})
  \item Můžete zkusit naprogramovat generování $GD(\mathcal{D})$.
    (hint: jednoduše bruteforce procházení všech podmnožin $R$) 
  \end {enumerate}
\end {exercise}

Nyní je však na místě se ptát, jak nám ale pomůže s výpočtem
$GD(\mathcal{D})$, když ji potřebujeme znát, abychom mohli spočítat
pevné body operátoru $\dots^{(\infty)}_{GD(\mathcal{D})}$. Trik je ve
využití $\subset$, které se vyskytuje u výpočtu, nebo-li k výpočtu je
potřeba znát pouze všechny ostré podmnožiny. Stačí tedy hledat pevné
body našeho operátoru v pořadí, které je obohacením relace
podmnožinovosti.

Nyní zavedeme úplné uspořádání na podmnožinách $R$ a k tomu účelu
seřadíme prvky $R$ a pro jednoduchost je stotožníme s čísly, tedy
$R=\{1,2,\dots,n\}$.

\begin {definition}
  Pro $A,B\subseteq R$ a $i\in R$ položíme $A<_iB$, pokud
  platí, že $i\in B\setminus A$ a $A\cap \{1,\dots,i-1\} = B\cap
  \{1,\dots,i-1\}$. Navíc $A<B$ pokud existuje $i\in R$ tak, že
  $A<_iB$.
\end {definition}
\begin {remark}
  Všimněme si, že pokud $A\subset B$, pak nutně $A<B$. Z toho pak
  plyne, že $\emptyset$ je vždy nejmenším prvkem v tomto uspořádání.
\end {remark}

Tato relace nám definuje lexikografické uspořádání, které je úplné,
tedy pro každé dvě $A,B\in R$ máme buď $A<B$ nebo $B<A$. V tomto
pořadí, pak budeme hledat pevné body operátoru
$\dots^{(\infty)}_{GD(\mathcal{D})}$.

\begin {example}
  Mějme $R= \{1,2,3,4,5,6\}$ a uvažujme množiny $$\{1\}, \{2\},
  \{2,3\}, \{3,4,5\}, \{3,6\}, \{1,4,5\}.$$

  Dle definice můžeme tyto množiny uspořádat následovně:
  $$ \{3,6\} <_4 \{3,4,5\} <_2 \{2\} <_3 \{2,3\} <_1 \{1\} <_4
  \{1,4,5\}.$$
\end {example}

Nyní definujeme operátor s jehož pomocí, pak budeme hledat pevné body
uzávěrového operátoru. Algoritmu, který pak vzejde z tvrzení o tomto
operátoru, se říká \texttt {NextClosure}.

\begin{remark}
  Připomeňme, že pevné body uzávěrového operátoru
  jsou ty množiny, které se operátorem zobrazí samy na sebe.
\end{remark}

\begin {definition}
  Pro $A\subseteq R$, $i\in R$ a uzávěrový operátor $c$ položíme
  $$ A\oplus_c i = c((A\cap \{1,\dots,i-1\}) \cup \{i\}).$$
\end {definition}

\begin {example}
  Mějme $R= \{a,b,c,d,e\}$ a uvažujme teorii
  \begin{align*}
    T = \{
    &\emptyset\Rightarrow \{e\},\\
    &\{a,b,e\}\Rightarrow \{a,b,d,e\},\\
    &\{c,d,e\}\Rightarrow \{a,b,c,d,e\}\}.
  \end{align*}

  Navíc budeme uvažovat lexikografické uspořádání na řetězcích,
  abychom mohli využít předchozí definici. To znamená, že $a<b<c<d<e$.
  Pak podle předchozí definice máme
  \begin{align*}
    \{c,d\} \oplus_{[\dots]_{T}} a
    &= [(\{c,d\}\cap\emptyset)\cup\{a\}]_{T} =
      [\{a\}]_{T} = \{a,e\} \\
    \{a,d\} \oplus_{[\dots]_{T}} c
    &= [(\{a,d\}\cap\{a,b\})\cup \{c\}]_{T} =
      [\{a,c\}]_{T} = \{a,c,e\} \\
    \{a,b,d\} \oplus_{[\dots]_{T}} d
    &= [(\{a,b,d\}\cap\{a,b,c\})\cup \{d\}]_{T} =
      [\{a,b,d\}]_{T} = \{a,b,d,e\} \\
    \emptyset \oplus_{[\dots]_{T}} d
    &= [(\emptyset\cap\{a,b,c\})\cup \{d\}]_{T} =
      [\{d\}]_{T} = \{d,e\} 
  \end{align*}
\end{example}

\begin {theorem}
  Nejmenší pevný bod $A^+$ uzávěrového operátoru $c$ na $R$, který je
  větší než $A\subseteq R$ vzhledem k uspořádání $<$, je dán předpisem
  $$ B^+ = B\oplus_c i,$$
  kde $i$ je největší číslo takové, že $B<_i B\oplus_c i$.
\end {theorem}

Důkaz přeskočíme a zaměříme se na význam předchozí věty. Abychom našli
všechny pevné body libovolného uzávěrového operátoru stačí začít s
prázdnou množinout a postupně hledat nejmenší větší uzávěry, což je
vlastně to, co dělá algoritmus \texttt{NextClosure}.

\begin{example}
  Uvažujme stejnou teorii a uspořádání jako v předchozím příkladu. Pak
  algoritmus \texttt{NextClosure} bude počítat následující:
  \begin{align*}
    \emptyset \oplus_{[\dots]_{T}} e
    &= [(\emptyset\cap\{a,b,c,d\})\cup\{e\}]_{T} =
      [\{e\}]_{T} = \{e\}\\
    &(\emptyset<_e \{e\})\\
    \texttt{NextClosure}&([\dots]_{T},\emptyset)=\{e\}\\    
    \{e\} \oplus_{[\dots]_{T}} d
    &= [(\{e\}\cap \{a,b,c\})\cup\{d\}]_{T} =
      [\{d\}]_{T} = \{d,e\}\\
    &(\{e\}<_d \{d,e\})\\
    \texttt{NextClosure}&([\dots]_{T},\{e\}=\{d,e\}\\    
    \{d,e\} \oplus_{[\dots]_{T}} c
    &= [(\{d,e\}\cap \{a,b\})\cup\{c\}]_{T} =
      [\{c\}]_{T} = \{c,e\}\\
    &(\{d,e\}<_c \{c,e\})\\
    \texttt{NextClosure}&([\dots]_{T},\{d,e\})=\{c,e\}\\
    \{c,e\} \oplus_{[\dots]_{T}} d
    &= [(\{c,e\}\cap \{a,b,c\})\cup\{d\}]_{T} =
      [\{c,d\}]_{T} = \{a,b,c,d,e\}\\
    &(\{c,e\}\not<_d \{a,b,c,d,e\})\\
    \{c,e\} \oplus_{[\dots]_{T}} b
    &= [(\{c,e\}\cap \{a\})\cup\{b\}]_{T} =
      [\{b\}]_{T} = \{b,e\}\\
    &(\{c,e\}<_b \{b,e\})\\
    \texttt{NextClosure}&([\dots]_{T},\{c,e\})=\{b,e\}\\
    \{b,e\} \oplus_{[\dots]_{T}} d
    &= [(\{b,e\}\cap \{a,b,c\})\cup\{d\}]_{T} =
      [\{b,d\}]_{T} = \{b,d,e\}\\
    &(\{b,e\}<_d \{d\})\\
    \texttt{NextClosure}&([\dots]_{T},\{b,e\})=\{d\}\\
    \{b,d,e\} \oplus_{[\dots]_{T}} c
    &= [(\{b,d,e\}\cap \{a,b\})\cup\{c\}]_{T} =
      [\{b,c\}]_{T} = \{b,c,e\}\\
    &(\{b,d,e\}<_c \{b,c,e\})\\
    \texttt{NextClosure}&([\dots]_{T},\{b,d,e\})=\{b,c,e\}\\
    \{b,c,e\} \oplus_{[\dots]_{T}} d
    &= [(\{b,c,e\}\cap \{a,b,c\})\cup\{d\}]_{T} =
      [\{b,c,d\}]_{T} = \{a,b,c,d,e\}\\
    &(\{b,c,e\}\not<_d \{a,b,c,d,e\})\\
    \{b,c,e\} \oplus_{[\dots]_{T}} a
    &= [(\{b,c,e\}\cap \emptyset)\cup\{a\}]_{T} =
      [\{a\}]_{T} = \{a,e\}\\
    &(\{b,c,e\}<_a \{a,e\})\\
    \texttt{NextClosure}&([\dots]_{T},\{b,c,e\})=\{a,e\}\\
    \vdots
  \end{align*}  
  Jak je vidět, vždy hledáme největší možný atribut, který lze
  \uv{přičíst} k množině tak, aby byl výsledek větší.

  Jako cvičení můžete pokračovat, dokud nedojdete k $R$.
\end{example}

My budeme chtít hledat pevné body operátoru
$\dots^{(\infty)}_{GD(\mathcal{D})}$. Jelikož ale předem neznáme
$GD(\mathcal D)$, musíme ji postupně budovat. Jak už jsme si řekli, k
výpočtu uzávěru potřebujeme vždy pouze FZ z $GD(\mathcal D)$ s
předpokladem ostře menším než daná množina. Navíc \texttt{NextClosure}
počítá uzávěry v pořadí obohacujícím podmnožinovost a tím pádem při
každém kroku už můžeme znát všechny tyto předpoklady.

Algoritmus \texttt{MinBase} budeme formulovat jako posloupnost teorií
a podmnožin $R$. Položíme $M_0=\emptyset$ a $T_0=\emptyset$ a další
prvky posloupnosti definujeme následovně:

\begin{align*}
  T_{n+1}=&
  \begin{cases}
    T_n,
    & \text{ pokud } M_n=C_{\mathcal D}(M_n) \text{ a}\\
    T_n\cup\{M_n\Rightarrow C_{\mathcal D}(M_n)\}
    & \text{ jinak.}
  \end{cases}\\
  M_{n+1} =&\texttt{ NextClosure}(\dots^{(\infty)}_{T_{n+1 }},M_n)
\end{align*}

Pro takhle sestavenou posloupnost platí, že je neklesající a navíc pro
$i\in\mathbf N_0$, pro které platí, že $M_i=R$ máme
$T_i=GD(\mathcal D)$.

\begin{example}
  Uvažujme relaci $D$ danou následující tabulkou:

  \begin{center}
  \begin{tabular}{c c c c c}
    a & b & c & d & e \\
    \hline
    1 & 2 & 2 & 2 & 1 \\
    3 & 2 & 5 & 1 & 1 \\
    1 & 1 & 1 & 1 & 1 \\
    2 & 1 & 4 & 3 & 1 \\
    2 & 1 & 3 & 3 & 1 
  \end{tabular}
\end{center}
Navíc uvažujme lexikografické uspořádání na atributech. Pak by
\texttt{MinBase} počítal následující:
  \begin{align*}
    T_0&=\emptyset\\
    M_0&=\emptyset\\
    T_1&=\{\emptyset\Rightarrow \{e\}\},
       &\text{ protože } \emptyset\neq C_{\mathcal D}(\emptyset)=\{e\}\\
    M_1&=\texttt{NextClosure}(\dots_{T_1}^{(\infty)},\emptyset)=\{e\}\\
    T_2&=\{\emptyset\Rightarrow \{e\}\},
       &\text{ protože } \{e\} = C_{\mathcal D}(\{e\})=\{e\}\\
    M_2&=\texttt{NextClosure}(\dots_{T_2}^{(\infty)}, \{e\})=\{d,e\}\\
    T_3&=\{\emptyset\Rightarrow \{e\}\},
       &\text{ protože } \{d,e\} = C_{\mathcal D}(\{d,e\})=\{d,e\}\\
    M_3&=\texttt{NextClosure}(\dots_{T_3}^{(\infty)}, \{d,e\})=\{c,e\}\\
    T_4&=\{\emptyset\Rightarrow \{e\}, \{c,e\}\Rightarrow \{a,b,c,d,e\}\},
       &\text{ protože } \{c,e\} \neq C_{\mathcal D}(\{c,e\})=\{a,b,c,d,e\}\\
    M_4&=\texttt{NextClosure}(\dots_{T_4}^{(\infty)}, \{c,e\})=\{b,e\}\\
    T_5&=\{\emptyset\Rightarrow \{e\}, \{c,e\}\Rightarrow \{a,b,c,d,e\}\},
       &\text{ protože } \{b,e\} = C_{\mathcal D}(\{b,e\})=\{b,e\}\\
    M_5&=\texttt{NextClosure}(\dots_{T_5}^{(\infty)}, \{b,e\})=\{b,d,e\} \\
    \vdots
  \end{align*}
    Jako cvičení můžete pokračovat, dokud nedojdete k $M_i=R$.
\end{example}

\begin {exercise}
  Naprogramujte algoritmus \texttt {MinBase}.
\end {exercise}
\end {document}
