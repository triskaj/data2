\documentclass{article}

\usepackage[czech]{babel}
\usepackage[utf8]{inputenc}
\usepackage{amsthm}
\usepackage{amsmath}

\newtheorem{definition}{Definice}
\newtheorem*{remark}{Poznámka}
\newtheorem*{corollary}{Důsledek}
\renewcommand*{\proofname}{Důkaz}
\newtheorem{theorem}{Věta}

\DeclareMathOperator{\Mod}{Mod}

\begin {document}

\section {Funkční závislosti stanovené z dat}
Pro danou relaci $\mathcal D$ chceme najít, co možná nejmenší teorii
$T$ tak, že $\mathcal D\models A \Rightarrow B$ právě, když $T\models
A\Rightarrow B$.

\begin {definition}
  Teorie $T$ se nazývá \underline {báze $\mathcal D$}, pokud pro
  každou $A\Rightarrow B$ platí $\mathcal D\models A \Rightarrow B$
  p.k. $T\models A\Rightarrow B$.
\end {definition}

\begin {remark}
  Bází $\mathcal D$ je obecně hodně. Např. pokud $T$ je báze $\mathcal
  D$ a navíc $T\models A\Rightarrow B$ pro nějakou $A\Rightarrow B
  \not\in T$, pak $T\cup \{A\Rightarrow B\}$ je opět báze.
\end {remark}

Z definice báze je zřejmé, že budeme-li mít dvě báze, budou mít stejné
sémantické důsledky, je proto žádoucí si takový jev pojmenovat.

\begin {definition}
  Teorie $T_1$ a $T_2$ jsou \underline {sémanticky ekvivalentní},
  značeno $T_1\equiv T_2$, jestliže pro libovolnou $A \Rightarrow B$
  platí $T_1 \models A\Rightarrow B$ právě, když $T_2 \models
  A\Rightarrow B$.
\end {definition}

Sémanticky ekvivalentní teorie, pak mají úzký vztah k pojmu model
teorie.

\begin {theorem}[o charakterizaci sémantické ekvivalence]
  Následující tvrzení jsou ekvivalentní:
  \begin {enumerate}
  \item $T_1\equiv T_2$,
  \item $\Mod(T_1)=\Mod(T_2)$,
  \item $\Mod_C(T_1)=\Mod_C(T_2)$,
  \item Pro libovolnou $A\subseteq R$ máme $[A]_{T_1}=[A]_{T_2}$.
  \end {enumerate}
\end {theorem}
\begin {proof}
  $1\Rightarrow 2$: Pro libovolnou $A\Rightarrow B$ máme
  $\Mod(T_1)\models A\Rightarrow B$ p.k. $T_1\models A\Rightarrow B$
  p.k. $T_2\models A\Rightarrow B$ p.k. $\Mod(T_1)\models A\Rightarrow
  B$. \\

  $2\Rightarrow 3$: Speciální případ.\\

  $3\Rightarrow 4$: Stejné uzávěrové systémy mají stejné uzávěrové
  operátory.\\

  $4\Rightarrow 1$: Pro libovolnou $A\Rightarrow B$ máme $T_1\models
  A\Rightarrow B$ p.k. $B\subseteq[A]_{T_1}=[A]_{T_2}$
  p.k. $T_2\models A\Rightarrow B$.
\end {proof}

\begin {corollary}
  Pokud jsou $T_1$ a $T_2$ báze $\mathcal D$, pak $T_1\equiv T_2$.
\end {corollary}

Pro snadnější charakterizaci pravdivosti v relaci si zavedeme
operátor, který bude fungovat podobně jako sémantický uzávěr u
teorie. Nejdříve však definujeme relaci na n-ticích.

\begin{definition}
  Pro $\mathcal{D}\subseteq \prod_{y\in R}D_y$ a $M\subseteq R$
  definujeme $E_{\mathcal{D}}: 2^R \rightarrow 2^{\mathcal{D} \times
    \mathcal{D}}$ předpisem
  $$E_{\mathcal{D}}(M)= \{\langle t, t'\rangle\in \mathcal
  D\times \mathcal D \mid t(M)=t'(M)\}.$$
\end{definition}

\begin{remark}
\begin {itemize}
\item
  Z definice $E_{\mathcal{D}}$ je hned zřejmé, že relace
  $E_{\mathcal{D}}(M)$ je ekvivalencí na $\mathcal{D}$ což také
  znamená, že můžeme udělat rozklad.

\item
  Význam vztahu $E_{\mathcal{D}}(M)\subseteq E_{\mathcal{D}}(M')$ je,
  že všechny dvojce n-tic, které se rovnají na $M$ se také rovnají na
  $M'$.

\item
  $E_{\mathcal{D}}$ je zřejmě antinonní, protože pokud $M_1\subseteq
  M_2$, všechny n-tice, které se rovnají na $M_2$ se tím spíš musí
  rovnat na $M_1$, tedy $E_{\mathcal{D}}(M_2)\subseteq
  E_{\mathcal{D}}(M_1)$.
\end {itemize}
\end{remark}

\begin {definition}
  Pro $\mathcal{D}\subseteq \prod_{y\in R}D_y$ a $M\subseteq R$
  definujeme $C_{\mathcal D}: 2^R \rightarrow 2^R$ předpisem
  $$C_{\mathcal{D}}(M) = \{y \in R \mid E_{\mathcal{D}}(M) \subseteq
  E_{\mathcal{D}}(\{y\})\}.$$
\end {definition}
\begin {remark}
  $C_{\mathcal D}(M)$ je vlastně množina atributů, na kterých jsou si
  rovny všechny dvojice n-tic z $\mathcal{D}$, které jsou si rovny na
  $M$. Důsledkem pak je, že $E_{\mathcal{D}}(M)\subseteq
  E_{\mathcal{D}}(C_{\mathcal D}(M))$. Důkaz je ponechán čtenáři.
\end {remark}

\begin{theorem}
 $C_{\mathcal{D}}$ je uzávěrový operátor na $R$.
\end{theorem}
\begin{proof}
  \begin{itemize}
  \item \emph {(extenzivita)}: Pokud $y \in M$, pak
    $E_{\mathcal{D}}(M) \subseteq E_{\mathcal{D}}(\{y\})$, protože
    pokud jsou si $t, t'$ rovny na všech atributech z $M$, tím spíš
    jsou si rovny na $y \in M$. Odtud dle definice $C_{\mathcal{D}}$
    dostáváme $y \in C_{\mathcal{D}}(M)$.

  \item \emph{(monotonie)}: Předpokládejme $M_1 \subseteq M_2$ a
    vezmeme $y \in C_{\mathcal{D}}(M_1)$. Poslední znamená, že
    $E_{\mathcal{D}}(M) \subseteq E_{\mathcal{D}}(\{y\})$. Z antitonie
    $E_{\mathcal{D}}$ dostáváme $E_{\mathcal{D}}(M_2) \subseteq
    E_{\mathcal{D}}(M_1) \subseteq E_{\mathcal{D}}(\{y\})$. Z definice
    $C_{\mathcal{D}}$ je $y \in C_{\mathcal{D}}(M_2)$.
    
  \item \emph{(idempotence)}: $C_{\mathcal{D}}(M) \subseteq
    C_{\mathcal{D}}(C_{\mathcal{D}}(M))$ platí z extenzivity. Pro
    obrácenou inkluzi máme následující posloupnost argumentů:

    \begin {align*}
      E_{\mathcal{D}}(M) &\subseteq E_{\mathcal{D}}(C_{\mathcal{D}}(M))
      \\
      \{y \in R \mid E_{\mathcal{D}}(C_{\mathcal{D}}(M)) \subseteq
      E_{\mathcal{D}}(\{y\})\} &\subseteq \{y \in R \mid
      E_{\mathcal{D}}(M) \subseteq E_{\mathcal{D}}(\{y\})\}
      \\
      C_{\mathcal{D}}(C_{\mathcal{D}}(M)) &\subseteq
      C_{\mathcal{D}}(M)
    \end {align*}
\end{itemize}
\end{proof}

\begin{theorem}[o charakterizaci pravdivosti]
Následující jsou ekvivalentní:
\begin{enumerate}
\item
$\mathcal{D} \models A \Rightarrow B$
\item
$E_{\mathcal{D}}(A) \subseteq E_{\mathcal{D}}(B)$
\item
$B \subseteq C_{\mathcal{D}}(A)$
\end{enumerate}
\end{theorem}

\begin{proof}
  $1\Rightarrow2$: Z definice $\mathcal{D} \models A \Rightarrow B$,
  pokud $t(A)=t'(A)$, pak $t(B)=t'(B)$, tzn. pokud $\langle t, t'
  \rangle \in E_{\mathcal{D}}(A)$, pak $\langle t, t' \rangle \in
  E_{\mathcal{D}}(B)$, tj. $E_{\mathcal{D}}(A) \subseteq
  E_{\mathcal{D}}(B)$.\\

  $2\Rightarrow3$: Předpokládejme $E_{\mathcal{D}}(A) \subseteq
  E_{\mathcal{D}}(B)$. Pro libovolný $y \in B$ pak z antitonie platí
  $E_{\mathcal{D}}(A) \subseteq E_{\mathcal{D}}(B) \subseteq
  E_{\mathcal{D}}(\{y\})$. To podle definice $C_{\mathcal{D}}$
  znamená, že $y \in C_{\mathcal{D}}(A)$, tj. $B \subseteq
  C_{\mathcal{D}}(A)$.\\

  $3\Rightarrow1$: Předpokládejme $B \subseteq C_{\mathcal{D}}(A)$.
  Dále mějme $t, t'\in \mathcal D$ takové, že $t(A)=t'(A)$ a vezmeme
  libovolné $y \in B$. Pak nutně $\langle t, t' \rangle \in
  E_{\mathcal{D}}(A)$ a navíc $E_{\mathcal{D}}(A) \subseteq
  E_{\mathcal{D}}(\{y\})$. Důsledkem je, že $t(y)=t'(y)$, tedy
  $t(B)=t'(B)$.
\end{proof}

\begin{theorem}[o charakterizaci báze]
  $T$ je báze $\mathcal{D}$ právě, když pro libovolné $A\subseteq R$
  máme $C_{\mathcal{D}}(A)=[A]_T$.
\end{theorem}

\begin{remark}
  Ekvivalentně $C_{\mathcal{D}}(M)=[M]_T = M^{\infty}_T = M^{+}_T$.
\end{remark}

\begin{proof}
  "$\Rightarrow$": Nechť $T$ je báze $\mathcal{D}$. Pak $[M]_T
  \subseteq [M]_T$ p.k. $T \models M \Rightarrow [M]_T$
  p.k. $\mathcal{D} \models M \Rightarrow [M]_T$ p.k. $[M]_T \subseteq
  C_{\mathcal{D}}(M)$. Obráceně máme $C_{\mathcal{D}}(M) \subseteq
  C_{\mathcal{D}}(M)$ p.k. $\mathcal{D} \models M \Rightarrow
  C_{\mathcal{D}}(M)$ p.k. $T \models M \Rightarrow
  C_{\mathcal{D}}(M)$ p.k. $C_{\mathcal{D}}(M) \subseteq [M]_T$.
  Dohromady tedy $C_{\mathcal{D}}(M) =[M]_T$. \\
  
  "$\Leftarrow$": Pokud $C_{\mathcal{D}}$ má stejné pevné body jako
  $[\dots]_T$, pak $\mathcal{D} \models A \Rightarrow B$ p.k. $B
  \subseteq C_{\mathcal{D}}(A) = [A]_T$ p.k. $T \models A \Rightarrow
  B$.
\end{proof}

Následující věta ukazuje, že pro libovolnou relaci existuje minimálně
jedna báze.

\begin{theorem}[o existenci báze]
  $T = \{A \Rightarrow C_{\mathcal{D}}(A) \mid A \subseteq R\}$ je
  báze $\mathcal{D}$.
\end{theorem}

\begin{proof}
  Dle předchozí věty stačí ověřit, že $C_{\mathcal{D}}(M) = [M]_T$ pro
  libovolnou $M$, tzn. ověřit, že $M = C_{\mathcal{D}}(M)$ právě když
  $M \in \mathcal{M}_T$.\\

  "$\Rightarrow$": Předpokládejme $M = C_{\mathcal{D}}(M)$ a vezmeme
  libovolnou $A \Rightarrow C_{\mathcal{D}}(A) \in T$ tak, že $A
  \subseteq M$. Z monotonie operátoru $C_{\mathcal{D}}$ dostaneme
  $C_{\mathcal{D}}(A) \subseteq C_{\mathcal{D}}(M) = M$. Dohromady
  tedy $\mathcal{D}_M \models A \Rightarrow C_{\mathcal{D}}(A)$
  p.k. $\mathcal{D}_M \in \Mod_C(T)$ p.k. $M \in \mathcal{M}_T$.\\

  "$\Leftarrow$": Předpokládejme, že $M \in \mathcal{M}_T$. To jest
  $\mathcal{D}_M \in \Mod_C(T)$. Speciálně pro $M \Rightarrow
  C_{\mathcal{D}}(M) \in T$ máme $\mathcal{D}_M \models M \Rightarrow
  C_{\mathcal{D}}(M)$. Odtud $C_{\mathcal{D}}(M) \subseteq M$ a
  přidáme-li extenzivitu $C_{\mathcal{D}}$ dostaneme
  $C_{\mathcal{D}}(M) = M$.
\end{proof}

Když už víme, že báze vždy existuje, přesuneme pozornost na její
velikost vzhledem k počtu FZ. Z předchozího textu vyplývá, že se
budeme snažit najít bázi ekvivalentní, ale co nejmenší.

\end {document}
